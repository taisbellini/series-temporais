\documentclass[aoas]{imsart}
%% LaTeX 2e style file for the processing of LaTeX2e files
%% of the following IMS/BS journals:
%%
%% - The Annals of Probability
%% - The Annals of Applied Probability
%% - The Annals of Statistics
%% - The Annals of Applied Statistics
%% - Statistical Science
%% - Probability Surveys
%% - Statistics Surveys
%% - Electronic Journal of Statistics
%% - Bernoulli
%% - Annales de l'Institut Henri Poincar\'e - Probabilit\'es et Statistiques
%% - Brazilian Journal of Probability and Statistics
%% - Bayesian Analysis
%%
%% - Institute of Mathematical Statistics, U.S.A.
%% - Bernoulli Society
%% - Institut Henry Poincare
%% - Brazilian Statistical Association
%% - International Society for Bayesian Analysis
%%
%% Macros written by Vytas Statulevicius, VTeX, Lithuania
%% Maintained by TeX group members, VTeX, Lithuania
%% for Institute of Mathematical Statistics, U.S.A.
%% Please submit bugs or your comments to latex-support@vtex.lt
%%
%% The original distribution is located at:
%% https://www.e-publications.org/ims/support

\RequirePackage{amsthm,amsmath,amsfonts,amssymb}
\RequirePackage[authoryear]{natbib}
\RequirePackage[colorlinks,citecolor=blue,urlcolor=blue]{hyperref}
\RequirePackage{graphicx}

% Added package
\usepackage[T1]{fontenc}
\usepackage[english]{babel}

% Pandoc syntax highlighting

% Pandoc citation processing


% Garantees bookdown compilation
%\usepackage{lmodern}

\makeatletter
\def\maxwidth{\ifdim\Gin@nat@width>\linewidth\linewidth\else\Gin@nat@width\fi}
\def\maxheight{\ifdim\Gin@nat@height>\textheight\textheight\else\Gin@nat@height\fi}
\makeatother
% Scale images if necessary, so that they will not overflow the page
% margins by default, and it is still possible to overwrite the defaults
% using explicit options in \includegraphics[width, height, ...]{}
\setkeys{Gin}{width=\maxwidth,height=\maxheight,keepaspectratio}
% Set default figure placement to htbp
\makeatletter
\def\fps@figure{htbp}
\makeatother
\setlength{\emergencystretch}{3em} % prevent overfull lines
\providecommand{\tightlist}{%
  \setlength{\itemsep}{0pt}\setlength{\parskip}{0pt}}

% alternative version to the shaded problem
\makeatletter
\@ifundefined{Shaded}{
}{\renewenvironment{Shaded}{\begin{kframe}}{\end{kframe}}}
\makeatother

\startlocaldefs
%%%%%%%%%%%%%%%%%%%%%%%%%%%%%%%%%%%%%%%%%%%%%
%                                          %%
% Uncomment next line to change            %%
% the type of equation numbering           %%
%                                          %%
%%%%%%%%%%%%%%%%%%%%%%%%%%%%%%%%%%%%%%%%%%%%%
\numberwithin{equation}{section}
%%%%%%%%%%%%%%%%%%%%%%%%%%%%%%%%%%%%%%%%%%%%%
%                                          %%
% For Axiom, Claim, Corollary, Hypothezis, %%
% Lemma, Theorem, Proposition              %%
% use \theoremstyle{plain}                 %%
%                                          %%
%%%%%%%%%%%%%%%%%%%%%%%%%%%%%%%%%%%%%%%%%%%%%
\theoremstyle{plain}
\newtheorem{axiom}{Axiom}
\newtheorem{claim}[axiom]{Claim}
\newtheorem{theorem}{Theorem}[section]
\newtheorem{lemma}[theorem]{Lemma}
%%%%%%%%%%%%%%%%%%%%%%%%%%%%%%%%%%%%%%%%%%%%%
%                                          %%
% For Assumption, Definition, Example,     %%
% Notation, Property, Remark, Fact         %%
% use \theoremstyle{remark}                %%
%                                          %%
%%%%%%%%%%%%%%%%%%%%%%%%%%%%%%%%%%%%%%%%%%%%%
\theoremstyle{remark}
\newtheorem{definition}[theorem]{Definition}
\newtheorem*{example}{Example}
\newtheorem*{fact}{Fact}
%%%%%%%%%%%%%%%%%%%%%%%%%%%%%%%%%%%%%%%%%%%%%
% Please put your definitions here:        %%
%%%%%%%%%%%%%%%%%%%%%%%%%%%%%%%%%%%%%%%%%%%%%
\endlocaldefs

% pandoc header

\begin{document}
\begin{frontmatter}
%%%%%%%%%%%%%%%%%%%%%%%%%%%%%%%%%%%%%%%%%%%%%%
%%                                          %%
%% Enter the title of your article here     %%
%%                                          %%
%%%%%%%%%%%%%%%%%%%%%%%%%%%%%%%%%%%%%%%%%%%%%%
\title{Quantile Autoregressive Distributed Lag model Global selection}
%\title{A sample article title with some additional note\thanksref{T1}}
\runtitle{}
%\thankstext{T1}{A sample of additional note to the title.}



\begin{aug}
%%%%%%%%%%%%%%%%%%%%%%%%%%%%%%%%%%%%%%%%%%%%%%
%%Only one address is permitted per author. %%
%%Only division, organization and e-mail is %%
%%included in the address.                  %%
%%Additional information can be included in %%
%%the Acknowledgments section if necessary. %%
%%%%%%%%%%%%%%%%%%%%%%%%%%%%%%%%%%%%%%%%%%%%%%

%% Example:
%%\author[A]{\fnms{First} \snm{Author}\ead[label=e1]{first@somewhere.com}},
%%\author[B]{\fnms{Second} \snm{Author}\ead[label=e2,mark]{second@somewhere.com}}
%%\and
%%\author[B]{\fnms{Third} \snm{Author}\ead[label=e3,mark]{third@somewhere.com}}

\author[A]{\fnms{Taís} \snm{Bellini}
  \ead[label=e1]{tais38@gmail.com}}
  

%%%%%%%%%%%%%%%%%%%%%%%%%%%%%%%%%%%%%%%%%%%%%%
%% Addresses                                %%
%%%%%%%%%%%%%%%%%%%%%%%%%%%%%%%%%%%%%%%%%%%%%%
%% Example:
%%\address[B]{Department,
%%University or Company Name,
%%\printead{e2,e3}}
\address[A]{Instituto de Matemática e Estatística, Universidade Federal
do Rio Grande do Sul,
  \printead{e1}}
\end{aug}

\begin{abstract}
The abstract should summarize the contents of the paper. It should be
clear, descriptive, self-explanatory and not longer than 200 words. It
should also be suitable for publication in abstracting services.
Formulas should be used as sparingly as possible within the abstract.
The abstract should not make reference to results, bibliography or
formulas in the body of the paper---it should be self-contained.

This is a sample input file. Comparing it with the output it generates
can show you how to produce a simple document of your own.
\end{abstract}


\begin{keyword}
\kwd{quantile regression}
\kwd{time series}
\kwd{Estimation}
\kwd{Regularization}
\kwd{adaLasso}
\end{keyword}

\end{frontmatter}


\hypertarget{introduction}{%
\section{Introduction}\label{introduction}}

Quantile regression, introduced in \citet{Koenker1978}, provides an
estimation of the conditional distribution of the response variable
\(Y\) on the vector of covariates \(\mathbf{X}\) at different quantile
levels, denoted here as \(\tau\in(0,1)\), thus, offering an
approximation for the whole conditional distribution. It offers a more
robust estimation for outliers, as opposed to classical linear
regression that only evaluates the conditional mean at a specific
location. \citep{Davino2014} In \citet{Koenker2005}, the Quantile
Autoregressive Model (QAR) is introduced, where the \(\tau\)-th
conditional quantile function of the response variable \(y_t\) is
expressed as a linear function of the lagged values of \(y\) and all of
the autoregressive coefficients can depend on \(\tau\). Quantile
regression estimation for other traditional time series models are
described in \citet{Koenker2018}, such as ARMA and ADL models.

In quantile regression, we get the conditional quantile function for a
determined \(\tau\). Therefore, for each desired quantile, there will be
a regression model and estimation. This brings complexity to certain
operations, since we need will have \(M\) different estimations, where
\(M\) is the number of quantiles we wish to evaluate. For example, if we
want to perform variable selection using regularization techniques, such
as LASSO or AdaLASSO, we might have different variables selected for
each \(\tau\). \citet{Sottile2020} proposes an approach for quantile
regression that estimates the coefficients for a given grid of
\(\tau\)'s in one single minimization problem. With this approach, it is
possible to perform a global selection of variables for a given grid of
quantiles. \citet{Sottile2020} studies global coefficient estimation and
variable selection in cross-sectional data using LASSO, demonstrating
being able to efficiently approximate the true model with high
probability.

In this work we propose a quantile regression of a time series model
with i.i.d. errors applying the global coefficient estimation approach
given in \citet{Sottile2020}. Furthermore, since we are in a time series
context, the variable selection penalization can be better explored, for
example, applying higher penalties to higher lags, as proposed in
\citet{Konzen2016}.

\hypertarget{quantile-autoregressive-distributed-lag-model-qadl}{%
\subsection{Quantile Autoregressive Distributed Lag Model
(QADL)}\label{quantile-autoregressive-distributed-lag-model-qadl}}

Consider an autoregressive-distributed lag model described by the
equation: \begin{equation}
y_t = c + \alpha_1y_{t-1} +  \cdots + \alpha_py_{t-p} + \theta_1\mathbf{x}_{t-1}^\intercal + \cdots + \theta_q\mathbf{x}_{t-q}^\intercal + \varepsilon_t 
\label{eq:adl}
\end{equation} where \(t = 1, ..., n\), \(y_t\) is the response
variable, \(y_{t-j}\) the lag of the response variable and
\(\mathbf{x}_{t-q}\) is the lagged covariates vector with dimension
\(d\). \(\varepsilon_t\) is a white noise.

As in \citet{Koenker2006}, we are interested in studying a class of
quantile autoregressive models which coefficients can be dependent of
\(\tau\). Therefore, lets consider the following process, with \({U_t}\)
as a sequence of i.i.d standard uniform random variables, \(c\),
\(\alpha\) and \(\theta\) as unknown functions
\([0,1] \rightarrow \mathbb{R}\) to be estimated: \begin{equation}
y_t = c(U_t) + \alpha_1(U_t)y_{t-1} +  \cdots + \alpha_p(U_t)y_{t-p} + \theta_1(U_t)\mathbf{x}_{t-1}^\intercal + \cdots + \theta_q(U_t)\mathbf{x}_{t-q}^\intercal
\label{eq:adltau}
\end{equation}

Then, given that for any monotone increasing function \(g\) and standard
uniform random variable \(U\) the following is true \[
Q_{g(U)}(\tau) = g(Q_U(\tau)) = g(\tau)
\] and assuming the right side of the equation \ref{eq:adltau} is
monotone and increasing on \(U_t\), we can say that the \(\tau_{th}\)
conditional quantile function of \(y_t\) is \begin{equation}
Q_{y_t}(\tau|\Im_t) =  c(\tau) + \alpha_1(\tau)y_{t-1} + \cdots + \alpha_p(\tau)y_{t-p} + \theta_1(\tau)\mathbf{x}_{t-1}^\intercal + \cdots + \theta_q(\tau)\mathbf{x}_{t-q}^\intercal
\label{eq:qadl}
\end{equation} where \(\Im_t\) is the \(\sigma\)-field generated by
\({y_s, s\leq t}\).

\hypertarget{global-coefficient-estimation-and-variable-selection}{%
\subsection{Global coefficient estimation and variable
selection}\label{global-coefficient-estimation-and-variable-selection}}

Let \(\mathbf{z_t}^\intercal\) be the covariates matrix with dimension
\(d\) and \(\beta(\tau)\) a vector of the coefficients that describe the
relationship between the covariates \(\mathbf{z}\) and the \(\tau\)-th
quantile of the response variable, \(\tau \in (0,1)\).

Then, equation \ref{eq:qadl} can be expressed in the form of
\begin{equation}
Q_{y_t}(\tau|\Im_t) = \mathbf{z_t}^\intercal\beta(\tau)
\label{eq:qadlred}
\end{equation}

Given the QADL model described above, \(\mathbf{z_t}^\intercal\) and
\(\beta(\tau)\) are, respectively, \[
\mathbf{z_t}^\intercal = (1, y_{t-1}, \cdots, y_{t-p}, \mathbf{x}_{t-1}, \cdots, \mathbf{x}_{t-p})
\] and \[
\beta(\tau)^\intercal = (c(\tau), \alpha_1(\tau), \cdots, \alpha_p(\tau), \theta_1(\tau), \cdots , \theta_q(\tau))
\]

In a standard quantile regression (QR) method, we would estimate
\(\beta\) for each \(\tau\) one at a time minimizing the expected value
of the check function \(\rho_\tau\): \[
\hat{\beta(\tau)} = \arg\min_b \frac{1}{n}\sum_{i=1}^n{\rho_\tau(y_i - z_i^\intercal b)}
\] where \(\rho_\tau(v) = v(\tau - \mathbb{I}_{[v<0]})\).
\citet{Frumento2016} suggested a different approach: modeling the
coefficient functions \(\beta(\tau)\) as parametric functions of the
order of the quantile. Consider \(\phi\) a vector of model parameters,
then we can describe the quantile function as: \begin{equation}
Q(\tau|\mathbf{z},\phi) = \mathbf{z}^\intercal\beta(\tau|\phi)
\label{eq:qrcm}
\end{equation}

As stated in \citet{Sottile2020}, this method improves the efficiency
and interpretation of the results, allowing us to maintain the quantile
regression structure in \ref{eq:qadlred} but modeling it parametrically.

To model \(\beta(\tau|\phi)\), a good practice is to use a flexible
model, such as a \(k\)th degree polynomial function: \[
\beta_j(\tau|\phi) = \phi_{j0} + \phi_{j1}\tau + \cdots + \phi_{jk}\tau^k
\] where \(j = 1, \cdots,q\) and \(q\) is the number of covariates of
the model. \(\phi\) will have dimensions \(q\)x\((k+1)\), so each
covariate has \(k+1\) associated parameters.

\hypertarget{estimation}{%
\subsection{Estimation}\label{estimation}}

Loss function

\hypertarget{variable-selection}{%
\subsection{Variable selection}\label{variable-selection}}

Loss function with penalty. AdaLasso. \citet{Konzen2016}.

\hypertarget{methodology}{%
\section{Methodology}\label{methodology}}

(Simulation part of Sottile + modifications)

\hypertarget{lists}{%
\subsection{Lists}\label{lists}}

The following is an example of an \emph{itemized} list, two levels deep.

\begin{itemize}
\tightlist
\item
  This is the first item of an itemized list. Each item in the list is
  marked with a ``tick.'' The document style determines what kind of
  tick mark is used.
\item
  This is the second item of the list. It contains another list nested
  inside it.

  \begin{itemize}
  \tightlist
  \item
    This is the first item of an itemized list that is nested within the
    itemized list.
  \item
    This is the second item of the inner list. \LaTeX\\
    allows you to nest lists deeper than you really should.
  \end{itemize}
\item
  This is the third item of the list.
\end{itemize}

The following is an example of an \emph{enumerated} list of one level.

\begin{longlist}
\item This is the first item of an enumerated list.
\item This is the second item of an enumerated list.
\end{longlist}

The following is an example of an \emph{enumerated} list, two levels
deep.

\begin{longlist}
\item[1.]
This is the first item of an enumerated list.  Each item
in the list is marked with a ``tick.''.  The document
style determines what kind of tick mark is used.
\item[2.]
This is the second item of the list.  It contains another
list nested inside of it.
\begin{longlist}
\item
This is the first item of an enumerated list that
is nested within.  
\item
This is the second item of the inner list.  \LaTeX\
allows you to nest lists deeper than you really should.
\end{longlist}
This is the rest of the second item of the outer list.
\item[3.]
This is the third item of the list.
\end{longlist}

\hypertarget{punctuation}{%
\subsection{Punctuation}\label{punctuation}}

Dashes come in three sizes: a hyphen, an intra-word dash like
``\(U\)-statistics'' or ``the time-homogeneous model''; a medium dash
(also called an ``en-dash'') for number ranges or between two equal
entities like ``1--2'' or ``Cauchy--Schwarz inequality''; and a
punctuation dash (also called an ``em-dash'') in place of a comma,
semicolon, colon or parentheses---like this.

Generating an ellipsis \ldots~with the right spacing around the periods
requires a special command.

\hypertarget{fonts}{%
\section{Fonts}\label{fonts}}

Please use text fonts in text mode, e.g.:

\begin{itemize}
\item[]\textrm{Roman}
\item[]\textit{Italic}
\item[]\textbf{Bold}
\item[]\textsc{Small Caps}
\item[]\textsf{Sans serif}
\item[]\texttt{Typewriter}
\end{itemize}

Please use mathematical fonts in mathematical mode, e.g.:

\begin{itemize}
\item[] $\mathrm{ABCabc123}$
\item[] $\mathit{ABCabc123}$
\item[] $\mathbf{ABCabc123}$
\item[] $\boldsymbol{ABCabc123\alpha\beta\gamma}$
\item[] $\mathcal{ABC}$
\item[] $\mathbb{ABC}$
\item[] $\mathsf{ABCabc123}$
\item[] $\mathtt{ABCabc123}$
\item[] $\mathfrak{ABCabc123}$
\end{itemize}

Note that \verb|\mathcal, \mathbb| belongs to capital letters-only font
typefaces.

\hypertarget{notes}{%
\section{Notes}\label{notes}}

Footnotes\footnote{This is an example of a footnote.} pose no
problem.\footnote{Note that footnote number is after punctuation.}

\hypertarget{quotations}{%
\section{Quotations}\label{quotations}}

Text is displayed by indenting it from the left margin. There are short
quotations

\begin{quote}
This is a short quotation. It consists of a single paragraph of text.
There is no paragraph indentation.
\end{quote}

and longer ones.

\begin{quotation}
This is a longer quotation. It consists of two paragraphs of text. The
beginning of each paragraph is indicated by an extra indentation.

This is the second paragraph of the quotation. It is just as dull as the
first paragraph.

\end{quotation}

\hypertarget{environments}{%
\section{Environments}\label{environments}}

\hypertarget{examples-for-plain-style-environments}{%
\subsection{\texorpdfstring{Examples for \emph{\texttt{plain}-style
environments}}{Examples for plain-style environments}}\label{examples-for-plain-style-environments}}

\begin{axiom}
\label{ax1} This is the body of Axiom \ref{ax1}.

\end{axiom}

\begin{proof}
This is the body of the proof of the axiom above.

\end{proof}

\begin{claim}
\label{cl1} This is the body of Claim \ref{cl1}. Claim \ref{cl1} is
numbered after Axiom \ref{ax1} because we used \verb|[axiom]| in
\verb|\newtheorem|.

\end{claim}

\begin{theorem}
\label{th1} This is the body of Theorem \ref{th1}. Theorem \ref{th1}
numbering is dependent on section because we used \verb|[section]| after
\verb|\newtheorem|.

\end{theorem}

\begin{theorem}[Title of the theorem]
\label{th2} This is the body of Theorem \ref{th2}. Theorem \ref{th2} has
additional title.

\end{theorem}

\begin{lemma}
\label{le1} This is the body of Lemma \ref{le1}. Lemma \ref{le1} is
numbered after Theorem \ref{th2} because we used \verb|[theorem]| in
\verb|\newtheorem|.

\end{lemma}

\begin{proof}[Proof of Lemma \ref{le1}]
This is the body of the proof of Lemma \ref{le1}.

\end{proof}

\hypertarget{examples-for-remark-style-environments}{%
\subsection{\texorpdfstring{Examples for \emph{\texttt{remark}}-style
environments}{Examples for remark-style environments}}\label{examples-for-remark-style-environments}}

\begin{definition}
\label{de1} This is the body of Definition \ref{de1}. Definition
\ref{de1} is numbered after Lemma \ref{le1} because we used
\verb|[theorem]| in \verb|\newtheorem|.

\end{definition}

\begin{example}
This is the body of the example. Example is unnumbered because we used
\verb|\newtheorem*| instead of \verb|\newtheorem|.

\end{example}

\begin{fact}
This is the body of the fact. Fact is unnumbered because we used
\verb|\newtheorem*| instead of \verb|\newtheorem|.

\end{fact}

\hypertarget{tables-and-figures}{%
\section{Tables and figures}\label{tables-and-figures}}

Cross-references to labeled tables: As you can see in Table\ref{tab:mtc}
and also in Table\ref{parset}.

\begin{table}

\caption{\label{tab:mtc}Table caption}
\centering
\begin{tabular}[t]{lrrrrrrrrrrr}
\hline
  & mpg & cyl & disp & hp & drat & wt & qsec & vs & am & gear & carb\\
\hline
Mazda RX4 & 21.0 & 6 & 160.0 & 110 & 3.90 & 2.620 & 16.46 & 0 & 1 & 4 & 4\\
Mazda RX4 Wag & 21.0 & 6 & 160.0 & 110 & 3.90 & 2.875 & 17.02 & 0 & 1 & 4 & 4\\
Datsun 710 & 22.8 & 4 & 108.0 & 93 & 3.85 & 2.320 & 18.61 & 1 & 1 & 4 & 1\\
Hornet 4 Drive & 21.4 & 6 & 258.0 & 110 & 3.08 & 3.215 & 19.44 & 1 & 0 & 3 & 1\\
Hornet Sportabout & 18.7 & 8 & 360.0 & 175 & 3.15 & 3.440 & 17.02 & 0 & 0 & 3 & 2\\
Valiant & 18.1 & 6 & 225.0 & 105 & 2.76 & 3.460 & 20.22 & 1 & 0 & 3 & 1\\
Duster 360 & 14.3 & 8 & 360.0 & 245 & 3.21 & 3.570 & 15.84 & 0 & 0 & 3 & 4\\
Merc 240D & 24.4 & 4 & 146.7 & 62 & 3.69 & 3.190 & 20.00 & 1 & 0 & 4 & 2\\
Merc 230 & 22.8 & 4 & 140.8 & 95 & 3.92 & 3.150 & 22.90 & 1 & 0 & 4 & 2\\
Merc 280 & 19.2 & 6 & 167.6 & 123 & 3.92 & 3.440 & 18.30 & 1 & 0 & 4 & 4\\
Merc 280C & 17.8 & 6 & 167.6 & 123 & 3.92 & 3.440 & 18.90 & 1 & 0 & 4 & 4\\
Merc 450SE & 16.4 & 8 & 275.8 & 180 & 3.07 & 4.070 & 17.40 & 0 & 0 & 3 & 3\\
Merc 450SL & 17.3 & 8 & 275.8 & 180 & 3.07 & 3.730 & 17.60 & 0 & 0 & 3 & 3\\
Merc 450SLC & 15.2 & 8 & 275.8 & 180 & 3.07 & 3.780 & 18.00 & 0 & 0 & 3 & 3\\
Cadillac Fleetwood & 10.4 & 8 & 472.0 & 205 & 2.93 & 5.250 & 17.98 & 0 & 0 & 3 & 4\\
Lincoln Continental & 10.4 & 8 & 460.0 & 215 & 3.00 & 5.424 & 17.82 & 0 & 0 & 3 & 4\\
Chrysler Imperial & 14.7 & 8 & 440.0 & 230 & 3.23 & 5.345 & 17.42 & 0 & 0 & 3 & 4\\
Fiat 128 & 32.4 & 4 & 78.7 & 66 & 4.08 & 2.200 & 19.47 & 1 & 1 & 4 & 1\\
Honda Civic & 30.4 & 4 & 75.7 & 52 & 4.93 & 1.615 & 18.52 & 1 & 1 & 4 & 2\\
Toyota Corolla & 33.9 & 4 & 71.1 & 65 & 4.22 & 1.835 & 19.90 & 1 & 1 & 4 & 1\\
Toyota Corona & 21.5 & 4 & 120.1 & 97 & 3.70 & 2.465 & 20.01 & 1 & 0 & 3 & 1\\
Dodge Challenger & 15.5 & 8 & 318.0 & 150 & 2.76 & 3.520 & 16.87 & 0 & 0 & 3 & 2\\
AMC Javelin & 15.2 & 8 & 304.0 & 150 & 3.15 & 3.435 & 17.30 & 0 & 0 & 3 & 2\\
Camaro Z28 & 13.3 & 8 & 350.0 & 245 & 3.73 & 3.840 & 15.41 & 0 & 0 & 3 & 4\\
Pontiac Firebird & 19.2 & 8 & 400.0 & 175 & 3.08 & 3.845 & 17.05 & 0 & 0 & 3 & 2\\
Fiat X1-9 & 27.3 & 4 & 79.0 & 66 & 4.08 & 1.935 & 18.90 & 1 & 1 & 4 & 1\\
Porsche 914-2 & 26.0 & 4 & 120.3 & 91 & 4.43 & 2.140 & 16.70 & 0 & 1 & 5 & 2\\
Lotus Europa & 30.4 & 4 & 95.1 & 113 & 3.77 & 1.513 & 16.90 & 1 & 1 & 5 & 2\\
Ford Pantera L & 15.8 & 8 & 351.0 & 264 & 4.22 & 3.170 & 14.50 & 0 & 1 & 5 & 4\\
Ferrari Dino & 19.7 & 6 & 145.0 & 175 & 3.62 & 2.770 & 15.50 & 0 & 1 & 5 & 6\\
Maserati Bora & 15.0 & 8 & 301.0 & 335 & 3.54 & 3.570 & 14.60 & 0 & 1 & 5 & 8\\
Volvo 142E & 21.4 & 4 & 121.0 & 109 & 4.11 & 2.780 & 18.60 & 1 & 1 & 4 & 2\\
\hline
\end{tabular}
\end{table}

\begin{table}
\caption{Sample posterior estimates for each model}
\label{parset}
%
\begin{tabular}{@{}lcrcrrr@{}}
\hline
&& & &\multicolumn{3}{c}{Quantile} \\
\cline{5-7}
Model &Parameter &
\multicolumn{1}{c}{Mean} &
Std. dev.&
\multicolumn{1}{c}{2.5\%} &
\multicolumn{1}{c}{50\%}&
\multicolumn{1}{c@{}}{97.5\%} \\
\hline
{Model 0} & $\beta_0$ & $-$12.29 & 2.29 & $-$18.04 & $-$11.99 & $-$8.56 \\
          & $\beta_1$  & 0.10   & 0.07 & $-$0.05  & 0.10   & 0.26  \\
          & $\beta_2$   & 0.01   & 0.09 & $-$0.22  & 0.02   & 0.16  \\[6pt]
{Model 1} & $\beta_0$   & $-$4.58  & 3.04 & $-$11.00 & $-$4.44  & 1.06  \\
          & $\beta_1$   & 0.79   & 0.21 & 0.38   & 0.78   & 1.20  \\
          & $\beta_2$   & $-$0.28  & 0.10 & $-$0.48  & $-$0.28  & $-$0.07 \\[6pt]
{Model 2} & $\beta_0$   & $-$11.85 & 2.24 & $-$17.34 & $-$11.60 & $-$7.85 \\
          & $\beta_1$   & 0.73   & 0.21 & 0.32   & 0.73   & 1.16  \\
          & $\beta_2$   & $-$0.60  & 0.14 & $-$0.88  & $-$0.60  & $-$0.34 \\
          & $\beta_3$   & 0.22   & 0.17 & $-$0.10  & 0.22   & 0.55  \\
\hline
\end{tabular}
%
\end{table}

\begin{figure}
\centering
\includegraphics{trabalho_series_files/figure-latex/unnamed-chunk-1-1.pdf}
\caption{Figure caption\label{penG}}
\end{figure}

Sample of cross-reference to figure. Figure\ref{penG} shows that it is
not easy to get something on paper.

\hypertarget{equations-and-the-like}{%
\section{Equations and the like}\label{equations-and-the-like}}

Two equations: \begin{equation}
    C_{s}  =  K_{M} \frac{\mu/\mu_{x}}{1-\mu/\mu_{x}} \label{ccs}
\end{equation} and \begin{equation}
    G = \frac{P_{\mathrm{opt}} - P_{\mathrm{ref}}}{P_{\mathrm{ref}}}  100(\%).
\end{equation}

Equation arrays: \begin{eqnarray}
  \frac{dS}{dt} & = & - \sigma X + s_{F} F,\\
  \frac{dX}{dt} & = &   \mu    X,\\
  \frac{dP}{dt} & = &   \pi    X - k_{h} P,\\
  \frac{dV}{dt} & = &   F.
\end{eqnarray} One long equation: \begin{eqnarray}
 \mu_{\text{normal}} & = & \mu_{x} \frac{C_{s}}{K_{x}C_{x}+C_{s}}  \nonumber\\
                     & = & \mu_{\text{normal}} - Y_{x/s}\bigl(1-H(C_{s})\bigr)(m_{s}+\pi /Y_{p/s})\\
                     & = & \mu_{\text{normal}}/Y_{x/s}+ H(C_{s}) (m_{s}+ \pi /Y_{p/s}).\nonumber
\end{eqnarray}

\begin{appendix}

\hypertarget{appn}{%
\section*{Title}\label{appn}}
\addcontentsline{toc}{section}{Title}

Appendices should be provided in \verb|{appendix}| environment, before
Acknowledgements.

If there is only one appendix, then please refer to it in text as
\ldots~in the \hyperref[appn]{Appendix}.

\end{appendix}

\begin{appendix}

\hypertarget{appA}{%
\section{Title of the first appendix}\label{appA}}

If there are more than one appendix, then please refer to it as
\ldots~in Appendix \ref{appA}, Appendix \ref{appB}, etc.

\hypertarget{appB}{%
\section{Title of the second appendix}\label{appB}}

\hypertarget{first-subsection-of-appendix}{%
\subsection{\texorpdfstring{First subsection of Appendix
\protect\ref{appB}}{First subsection of Appendix }}\label{first-subsection-of-appendix}}

Use the standard \LaTeX~commands for headings in \verb|{appendix}|.
Headings and other objects will be numbered automatically.

\begin{equation}
\mathcal{P}=(j_{k,1},j_{k,2},\dots,j_{k,m(k)}). \label{path}
\end{equation}

Sample of cross-reference to the formula (\ref{path}) in Appendix
\ref{appB}.

\end{appendix}

\hypertarget{acknowledgements}{%
\subsection*{Acknowledgements}\label{acknowledgements}}
\addcontentsline{toc}{subsection}{Acknowledgements}

The authors would like to thank the anonymous referees, an Associate
Editor and the Editor for their constructive comments that improved the
quality of this paper.

The first author was supported by NSF Grant DMS-??-??????.

The second author was supported in part by NIH Grant ???????????.

\begin{supplement}
\stitle{Title of Supplement A}
\sdescription{Short description of Supplement A.}
\end{supplement}
\begin{supplement}
\stitle{Title of Supplement B}
\sdescription{Short description of Supplement B.}
\end{supplement}

\bibliographystyle{imsart-nameyear}
\bibliography{ims.bib}


\end{document}
